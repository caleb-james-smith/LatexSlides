\documentclass[10pt,xcolor=svgnames,fleqn,aspectratio=169]{beamer}

\usepackage{tikz}
\usepackage{hyperref}
\usepackage{xcolor,colortbl}
\usepackage[absolute,overlay]{textpos}
\usepackage{graphicx}
\usepackage{cancel}

\def\blue{\color{DodgerBlue}}
\def\green{\color{Green}}
\def\red{\color{Red}}
\def\Orchid{\color{Orchid}}
\definecolor{DodgerBlueDark}{HTML}{1873cc}

\mode<presentation>
\usetheme{Copenhagen}
\usecolortheme[named=DodgerBlueDark]{structure}
%\useoutertheme{infolines}
\useoutertheme[footline=authortitle]{miniframes}
%\setbeamertemplate{headline}[infolines theme]
\setbeamertemplate{headline}[default]
\setbeamertemplate{navigation symbols}{}
\setbeamertemplate{footline}[miniframes theme]
\setbeamertemplate{footline}[page number]
\setbeamertemplate{itemize items}[triangle]

\newcommand{\dm}{$\Delta m$~}
\newcommand{\dphi}{$\Delta \phi$~}
\newcommand{\pt}{$p_{T}$~}
\newcommand{\texeta}{$\eta$~}

\title{Z to Invisible: Data and MC Stack Plots}

\author{\textcolor{DodgerBlueDark}{\bf C.~Smith\inst{1}}}
\institute{\inst{1} Baylor}
\date{\today}

\renewcommand{\arraystretch}{1.2}
\begin{document}

\begin{frame}[plain]
\maketitle
\end{frame}

\begin{frame}{Summary}
Data vs MC in Photon CR
\begin{itemize}
\item Low \dm and High \dm selections
\item Jet $p_T > 20$ GeV requirement
\begin{itemize}
\item Nb, mtb, and ptb
\end{itemize}
\item Jet $p_T > 20, 30, 40$ GeV requirements
\begin{itemize}
\item nJets, dPhi, HT, sMET, JetID, and HEMVeto
\end{itemize}
\item Photon selection has \pt and \texeta requirements, as well as loose, medium, and tight IDs
\end{itemize}
\end{frame}

\begin{frame}{Changes}
Recent Changes
\begin{itemize}
\item Remove QCD 100 to 200 HT sample.
\item Split 2017 into periods B-E and period F and use different pile-up weights.
\item Try jet pt cuts 20, 30, 40 and photon IDs loose, medium, and tight.
\item Calculate JetID and HEMVeto using different jet pt cuts.
\item Always use loose photons for modified MET and jet cleaning, even when selecting one medium or tight photon.
\end{itemize}
\end{frame}

%%%%%%%%%%%%%%%%%%%%%%%%%%%%%%%%%%%%%%%%%%

\input{stack_snippet.tex}

%%%%%%%%%%%%%%%%%%%%%%%%%%%%%%%%%%%%%%%%%%

\end{document}